\documentclass[letterpaper,10pt,notitlepage,fleqn]{article}

%\usepackage{nopageno} %gets rid of page numbers
\usepackage{alltt}                                           
\usepackage{float}
\usepackage{color}
\usepackage{indentfirst}
\usepackage{url}
\usepackage{balance}
\usepackage[TABBOTCAP, tight]{subfigure}
\usepackage{enumitem}
\usepackage{pstricks, pst-node}
\usepackage{geometry}
\geometry{textheight=9in, textwidth=6.5in} %sets 1" margins 
\newcommand{\cred}[1]{{\color{red}#1}} %command to change font to red
\newcommand{\cblue}[1]{{\color{blue}#1}} % ...blue
\usepackage{hyperref}
\usepackage{textcomp}
\usepackage{listings}
\usepackage{graphicx}

\def\name{Sam Quinn}

\parindent = 0.4444 in
\parskip = 0.2 in

\begin{document}
\begin{titlepage}
\vspace*{\fill}

\newcommand{\HRule}{\rule{\linewidth}{0.5mm}} % Defines a new command for the horizontal lines, change thickness here

\center % Center everything on the page

%----------------------------------------------------------------------------------------
%TITLE SECTI   ON
%----------------------------------------------------------------------------------------

%\includegraphics[scale=.5]{image.eps}
\HRule \\[0.4cm]
{ \huge \bfseries CS427 Project}\\[0.4cm] % Title of your document

%----------------------------------------------------------------------------------------
%HEADING SECTIONS
%----------------------------------------------------------------------------------------

\textsc{\LARGE SOCat Cryptography Analisis}\\[0.5cm] % Name of your university/college
\textsc{\Large CS427 Project}\\[0.5cm] % Major heading such as course name
\textsc{\large Winter 2016}\\[0.5cm] % Minor heading such as course title


\HRule \\[1.5cm]
%----------------------------------------------------------------------------------------
%AUTHOR SECTION
%------------------------------------ ----------------------------------------------------

\begin{minipage}{0.4\textwidth}
\begin{flushleft} \large
\emph{Student:}\\
        \textbf{Sam \textsc{Quinn}} \\ % Your name
        {\small Quinnsa@Oregonstat.edu}
        \end{flushleft}
        \end{minipage}
        ~
        \begin{minipage}{0.4\textwidth}
        \begin{flushright} \large
        \emph{Professor:} \\
            \textbf{Michael \textsc{Rosulek}} \\ % Supervisor's Name
            {\small rosulekm@eecs.oregonstate.edu}
            \end{flushright}
            \end{minipage}\\[3cm]

                %----------------------------------------------------------------------------------------
                %DATE SECTION
                %-----------------    -----------------------------------------------------------------------

{\large \today}\\[3cm] % Date, change the \today to a set date if you want to be precise

%----------------------------------------------------------------------------------------
%LOGO SECTION
%------   ----------------------------------------------------------------------------------

%\includegraphics{Logo}\\[1cm] % Include a department/university logo - this will require the graphicx package

%----------------------------------------------------------------------------------------

\vfill % Fill the rest of the page with whitespace



\end{titlepage}

\tableofcontents
\newpage

\section{Introduction}
Within Unix operating systems everything is considered a file, “socat” is a means of outputting or concatenating output to a socket. Socat can relay data in a bidirectional manner between two independent data channels. Socat can transmit to a file, pipe, device, or as socket (IP4, IP6, UDP, TCP) all encrypted over SSL. SSL is mandatory here since anytime you send data any eavesdroppers can view your data in plaintext, unless however it is encrypted. 
\\
Socat can be used as a TCP port forwarder essentially socks proxying TCP traffic to another remote system in a secure manner. There for all of your outbound traffic will pass through the secure socket tunnel to the remote system and from there fulfill the desired request. 
\\
As you can see the importance of security and encrypted data transfers in this application. As I mentioned earlier all network data traffic is susceptible to eavesdroppers, the protection comes from the data transmitted being encrypted where the adversary would not be able to understand the data that is being transferred. The entire security of the data transmitted over the wire is based on the encryption model that the developers of the application have implemented. Oddly enough many developers do not specialize in cryptography and many so called “secure” applications are inherently very insecure. This was the fact for Socat as well.
\\
Socat secures network transmission with the extremely popular OpenSSL. OpenSSL is infact still considered very secure, Socats problems were a product of their own misuse of the secure libraries within OpenSSL as defined in the following section.

\section{Problem Defined}
Socat’s implemented a hybrid encryption scheme to securely transfer data among peers. The symmetric keys are obtained using Diffie Hellman key exchange. 
Diffie Hellman protocol needs to publicly share two variables g and p to function correctly. The variable p is very large, at least 512 bits in length. The p value will be used as the modulus for the combined shared key after the two parties raise the g to their private keys. When generating p and g  you want to choose good numbers but most importantly both these numbers MUST be prime, which is where Socat went wrong. 
Socat for a full year had a p value that was not prime.  
\\
Their exact value was 11435638110073884015312138951374632602058080675070521707579703088370446597672067452229024566834732449017970455481029703480957707976441965258194321262569523
isprime(11435638110073884015312138951374632602058080675070521707579703088370446597672067452229024566834732449017970455481029703480957707976441965258194321262569523)
%1 = 1
\\
143319364394905942617148968085785991039146683740268996579566827015580969124702493833109074343879894586653465192222251909074832038151585448034731101690454685781999248641772509287801359980318348021809541131200479989220793925941518568143721972993251823166164933334796625008174851430377966394594186901123322297453
\\
27788893276069724796504555675597658900595616769773727063231875314156885361379100133264804184710789407128574011804155595735704837674243828066040543912171576627544718762752948158991754559261759162739343094515270757451837630913502740443023902769553802723685440839891240497710460941757089246131322686180648463540974702859210630184042730717698427486397505787974799692901205514386555272667298045803284972074823213104807295638814082142694729938965663710648170010420323923305528998108799706139846097432481556448740855888110797022123731105964852194684036975049177742094726795060211226322344210328442014189175085444396370522979
\\
So this is where it gets a little fishy for me, as we can see above that the first smallest value is not prime. Then on January 4 2015 Zhigang Wang, an Oracle Solution Specialist reported the bug and a submitted the patch. This was mentioned in the commit message made by Gerhard Rieger the maintainer and lead developer.

\section{Why the problem is bad}
supa bad!
\subsection{Backdoor Possibilitties}
Oh yeah
\section{How they fixed the problem}
better key
\section{Conclusion}
The end
\subsection{What I learned}
A lot
\end{document}
