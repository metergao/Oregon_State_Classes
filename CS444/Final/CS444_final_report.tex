\documentclass[letterpaper,10pt,notitlepage,fleqn]{article}

\usepackage{nopageno} %gets rid of page numbers
\usepackage{alltt}                                           
\usepackage{float}
\usepackage{color}
\usepackage{url}
\usepackage{balance}
\usepackage[TABBOTCAP, tight]{subfigure}
\usepackage{enumitem}
\usepackage{pstricks, pst-node}
\usepackage{geometry}
\geometry{textheight=9in, textwidth=6.5in} %sets 1" margins 
\newcommand{\cred}[1]{{\color{red}#1}} %command to change font to red
\newcommand{\cblue}[1]{{\color{blue}#1}} % ...blue
\usepackage{hyperref}
\usepackage{textcomp}
\usepackage{listings}
\usepackage{graphicx}
\usepackage{tikz}
\usetikzlibrary{shapes,arrows}
\usepackage{natbib}
\usepackage{indentfirst}

\def\name{Sam Quinn}

\parindent = 1 cm 
\parskip = 0.1 in

\definecolor{dkgreen}{rgb}{0,0.6,0}
\definecolor{gray}{rgb}{0.5,0.5,0.5}
\definecolor{mauve}{rgb}{0.58,0,0.82}
\lstset{frame=tb,
  language=c,
  aboveskip=3mm,
  belowskip=3mm,
  showstringspaces=false,
  columns=flexible,
  basicstyle={\small\ttfamily},
  numbers=none,
  numberstyle=\tiny\color{gray},
  keywordstyle=\color{blue},
  commentstyle=\color{dkgreen},
  stringstyle=\color{mauve},
  breaklines=true,
  breakatwhitespace=true,
  tabsize=3
}

\title{Final}
\author{Sam Quinn}

\begin{document}
\maketitle
\hrule

\section{CPU Scheduling}
Because operating systems run multiple processes concurrently the kernel needs 
to ensure each process is scheduled for execution. While many processes can run 
together not every process can run at once. This is where the CPU Scheduler comes 
into play. To understand how the kernel schedules these processes it is helpful to 
know the different types of ``Multitasking'' \textit{cooperative Multitasking} and 
\textit{preemptive multitasking}. Most modern operating systems implement a preemptive 
algorithm for scheduling which means an external entity dictates what process should 
be scheduled next. I will not be covering \textit{cooperative Multitasking} in 
which the processes schedule themselves. 

\subsection{Scheduling Algorithms}
As you can imagine allowing each process to execute can be a hard task. Even with 
modern systems the kernel has to be wary of how it allocates its resources. One 
of the main concepts behind how the scheduling algorithms work is that when a 
process does not have any work to be done that process can go into a \textit{block} 
or \textit{sleep} state to allow for other process that have work to be done to execute.

\subsubsection*{Linux} 
In earlier versions of Linux the kernel implemented an \textit{O(1)} algorithm that 
as the name suggests, ran at constant time. This was said to be a great algorithm 
for larger systems but failed to perform well on I/O heavy devices. During the rise of
Linux personal computers the \textit{CFS} (Completely fair scheduler) was created 
to handle blocking I/O operations better and is the default is today~\cite{LKD3}.

The Linux scheduler, like many other Linux features, can be customized by selecting 
different algorithms. Linux allows changing of the CPU scheduling algorithm either system 
wide or per process basis. The is done with the implementation of a \textit{Scheduler 
Classes}. \textit{Scheduler Classes} allows different classes of processes to utilize 
different algorithms with different policies concurrently~\cite{LKD3}.

The way the processes are scheduled by the selected algorithm is based on their priority. In Linux there are 
two types of priorities we need to worry about \textit{real-time} priority and \textit{nice}
priority. The \textit{nice} value is supposed to essentially rank the process worthiness 
for processor time. Linux has sort of a fun way of implementing the \textit{nice} 
priority system. The scale ranges from [-20 to 19] with the highest priority being -20, and
19 being the lowest or ``the most nice'' to other processes. The \textit{real-time} priorities are implemented opposing the 
nice implementation with a scale from [0 to 99] with 99 being the highest priority.
In normal configurations \textit{real-time} processes have a greater priority over 
the system than normal processes~\cite{LKD3}. 

\subsubsection*{Windows} 
Windows also schedules processes with a priority base preemptive scheduling algorithm 
with a few differences. Windows has had the same reliable scheduling algorithm since
the first release of Windows NT with scalability improvement with each release thereafter.
The Windows scheduler works by trying to equally share the processor or processors 
only among other processes with the same priority. This causes the problem that 
the Windows scheduler doesn't work well with different higher-level requirements 
like distribution of CPU time with multiple users in a terminal environment. The 
new scheduler named \textit{DFSS} (Distributed Fair Share Scheduling) solved this 
problem later on. Windows determines what algorithm to use during 
the \textit{PsBootPhaseComplete} final post-boot stage~\cite{WI16}. 

Like mentioned before, Windows shares the use of priories with Linux but the values 
are implemented differently. Windows seems to have a more understandable priority 
value system than Linux's implementation. Windows separates scale of 32 values 
into two groups, real-time levels and variable levels. The real-time values occupy 
the upper 15 levels of the priority [16 to 31]. The variable levels range from [1 to 15] 
reserving the priority 0 for the system which is one per system. The Windows priority 
levels go in one uniform direction with the higher the priority value the higher 
the priority~\cite{WI16}.

\subsection*{Context Switches}
When a new process has a turn in the queue or if their time slice expired the kernel 
initial a \textit{context switch}. A \textit{context switch} is the live swapping 
of one runnable task to another. Change a running process out has two aspects that
must be done. First is that they need to map their own virtual memory. Second is 
reimplement the processes stack information and registers. This context switch takes 
place every time a new process begins or resumes. 

\subsubsection*{Linux} 
In the Linux system, context switches are executed by the \textbf{context\_switch()} 
system call. Context switches are initiated by the Process scheduling algorithm 
with the \textbf{schedule()} call. The \textbf{context\_switch()} system call, 
like mentioned above, needs to remap the virtual memory and copy the stack information. 
To remap the virtual memory the \textbf{context\_switch()} system call issues a 
call for \textbf{switch\_mn()}, that removes the old processes mappings and replaces 
them with its own. Copying of the stack information is completed with a different 
call named \textbf{switch\_to()} which both saves and replaces the stack information 
when a context switch occurs~\cite{LKD3}. 
\begin{lstlisting}
    
context_switch(struct rq *rq, struct task_struct *prev,struct task_struct *next){
    struct mm_struct *mm, *oldmm;
    
    prepare_task_switch(rq, prev, next);
    my_trace_sched_switch(rq, prev, next);
    
    mm = next->mm;
    oldmm = prev->active_mm;
...
\end{lstlisting}

\subsubsection*{Windows}
Windows also needs to perform the same general principle of switching execution 
to a separate process. The Windows kernel must first save the information of the 
old thread by creating a stack pointer to the underlaying \textit{KTHREAD} structure. 
Once the old stack is safe the new process then pushes its own data on the current 
running process stack. If the new process does not share the same address space as 
the last running process it must also load the its page table. The new  process
copies the old address and stores it for latter while loading its new page table 
into one of the processors registers~\cite{WI16}.

\subsection*{Comparison}
\noindent \textbf{Similarities}\\
\begin{itemize}
    \item Have options to change algorithms.
    \item Each process has a priority
    \item Both have different priorities for real-time and normal processes.
    \item Context switches are fundamentally the same.
\end{itemize}
\noindent \textbf{Differences}\\
\begin{itemize}
    \item Windows Windows implements only one priority value for both real-time and 
    variable processes.
    \item Linux's priority of nice values is backwards from Windows.

\end{itemize}


\section{Processes and Threads}
To compute efficiently one very important concept to understand is how a programs operates in 
a parallel environment. There are two ways a program may run in parallel, 
threads and processes. While both treads and processes accomplish the same task of 
running in parallel each have their correct application. One of the main difference 
between these two and should be taken into consideration when choosing weather 
you should use threads or processes, is that threads 
share the same address space while each process has its own address space~\cite{PI}.

\subsection{Processes}
Processes are most useful when parallelization is required but sharing of system resources 
is nonessential. Processes often times do not share the same address space and are independent 
from all other process. The creation of a process can be an expensive task since 
each process that is created has to allocate its own address space and force the 
CPU to execute a context switch. 

\subsubsection*{Linux}
To create a process in a Linux operating system the \textbf{fork()} system call is 
used. When the \textbf{fork()} system call is called it creates a child process from 
the currently executing process which is refereed to as the parent. It is common to 
initiate a new program after the \textbf{fork()} command. To create a new namespace 
and begin executing a new program in Linux the \textbf{exec()} system call is used. 
If the parent process relies on the child process it can inquire about the child's 
status by using the \textbf{wait()} system call. Child processes when completed exit 
using the \textbf{exit()} system call, however, these processes are not completely 
destroyed from the system quite yet. When a child process calls \textbf{exit()} it is put 
into a ``Zombie'' state until the parent calls \textbf{wait()}~\cite{LKD3}.

Underneath the hood in Linux each process has its own ``Process Descriptor'' which 
keeps all of the processes information stored at the end of its stack, which is linked 
to a system wide doubly linked list. When \textbf{fork()} is called it initiates 
the \textbf{clone()} system call which does just what you would expect clones the 
parent process. The child and the parent at this point share the same copy of everything 
including their process descriptors. At this stage of process creation the \textbf{clone()} 
system call will differentiate the child process from the parent and pass required 
information if the child would like shared address spaces, open files, and 
namespaces. These flags are really the only thing that differentiates the creation 
of a process from that of a Linux thread~\cite{LKD3}.
\begin{lstlisting}
...
    char *stack;
    char *stacktop;
    pid_t pid;

    // Processes
    pid = clone(childFunc, stackTop, CLONE_NEWUTS | SIGCHLD, argv[1]);    
    if (pid == -1)
        errExit("clone");
    printf("clone() returned %ld\n", (long) pid);

    // Threads
    pid = clone(childFunc, staktop, CLONE_VM | CLONE_FS | CLONE_FILES | CLONE_SIGHAND);
    if (pid == -1)
        errExit("clone");
    printf("clone() returned %ld\n", (long) pid);

...
\end{lstlisting}
\subsubsection*{Windows}
Within a Windows operating system the main principles of threads and processes 
are preserved with process having their own address spaces and threads sharing 
the parents address space, however, Microsoft Windows does it slightly different. 
Windows uses the \textbf{CreateProcess()} system call which functions similar to 
the Linux \textbf{fork()} implementation. All of the information needed to execute 
a program is passed in to the \textbf{CreateProcess()} function. Since Windows does 
not clone the parent process like Linux does the \textbf{CreateProcess()} function 
requires many more parameters to achieve the same effect but potentially has more 
control of the child process~\cite{WI16}. 
\begin{lstlisting}
...
// Start the child process. 
    if( !CreateProcess( NULL,   // No module name (use command line)
        argv[1],        // Command line
        NULL,           // Process handle not inheritable
        NULL,           // Thread handle not inheritable
        FALSE,          // Set handle inheritance to FALSE
        0,              // No creation flags
        NULL,           // Use parent's environment block
        NULL,           // Use parent's starting directory 
        &si,            // Pointer to STARTUPINFO structure
        &pi )           // Pointer to PROCESS_INFORMATION structure
    ) 
    {
        printf( "CreateProcess failed (%d).\n", GetLastError() );
        return;
    }
...
\end{lstlisting}
\subsection{Threads}
Threads function very similar to that of processes in regards to performance but 
with the added benefit of being able to share the same namespace as the parent. 
Threads are essentially ``light weight'' process since they do not require their 
own address space and namespace it is more memory efficient, however, threads can 
not have threads of their own. 
\subsubsection*{Linux}
The Linux kernel views each thread as a unique process that merely shares resources 
of another process. Every data structure that is used for a process is also used 
for a thread within a Linux system. While Microsoft Windows has specific kernel 
support for threads to allow threads to be simple forms of processes, Linux on the 
other hand does not implement any thread improvements since Linux processes themselves 
are already considered ``Light Weight''. 
The Linux implementation of a thread is the same as that of a process and are created 
the same way. Linux creates a thread with the same \textbf{clone()} system call 
as with a process, with the exception of a few flags. 
The flags passed to the \textbf{clone()} system call instruct the kernel to share 
the address space, filesytem resources, file descriptors, and signal handlers to 
the newly created thread~\cite{LKD3}. 

\subsubsection*{Windows}
Like mentioned above Windows has specific implementations to create a thread unlike 
the universal \textbf{clone()} system call that Linux uses. To create a thread in 
Windows the \textbf{CreateThread()} is used. Each thread created in Windows shares 
the thread context of the main thread (parent process). The thread context includes 
the thread's set of machine registers, the kernel stack, a thread environment block, 
and a user stack in the address space. Windows threads also acquire the same security 
context as that of their parent unless specifically setup to receive otherwise~\cite{WI16}. 
\begin{lstlisting}
// Create the thread to begin execution on its own.

        hThreadArray[i] = CreateThread( 
            NULL,                   // default security attributes
            0,                      // use default stack size  
            MyThreadFunction,       // thread function name
            pDataArray[i],          // argument to thread function 
            0,                      // use default creation flags 
            &dwThreadIdArray[i]);   // returns the thread identifier 
\end{lstlisting}


\subsection*{Comparison}
\noindent \textbf{Similarities}
\begin{itemize}
    \item Support both Threads, and Processes.
    \item Processes do not share an address space while threads do.
\end{itemize}
\noindent \textbf{Differences}
\begin{itemize}
    \item Linux uses the \textbf{clone()} call to interact with the kernel directly 
    where as Windows uses the \textbf{CreateProcess()} which must go through the 
    Windows API first.
    \item Windows threads are viewed as threads while Linux threads are just process
    with shared memory.
\end{itemize}

\section{File Systems}
File systems are the fundamental principle behind how data is transfered in I/O situations. 
Every file store in a computer is stored and retrieved via a file system. Without 
a file system all of the data on the disk would be a pile of bytes unknowing where one 
file ends and another begins. Microsoft Windows has a very different way of approaching 
filesystems than the Linux implementation.

\subsection{Filesystem Support}
For each filesystem there needs to be a way for the data structures to be written 
and read from the physical volume. This is where the filesystem drivers come 
in. Both Linux and Windows have the need for Drivers but the way they are implemented 
with in the operating systems are very different.

\subsubsection*{Linux} 
Linux supports dozens of different filesystems with the most common being Minux, Ext[2,3,4] 
and JFS. Linux has one of the most versatile filesystem support all because of the 
\textit{VFS} (Virtual Filesystem). The \textit{VFS} creates the abstraction from 
real underlaying file I/O operations. With all of the I/O operations flowing through
the \textit{VFS} for Linux to support different filesystem all that is needed is 
a driver to translates standardized \textit{VFS} data operations to the current 
filesystem operations. This allows the Linux system to perform I/O operations without 
even needing to know what the actual filesystem of the disk is.~\cite{LKD3}.
\begin{lstlisting}
   ...
   #include <linux/fs.h>

   // Registering a Filesystem
   extern int register_filesystem(struct file_system_type *);
   // Unregistering a Filesystem
   extern int unregister_filesystem(struct file_system *);
   ...
\end{lstlisting}

\subsubsection*{Windows} 
Windows on the other hand only supports a handful of filesystems. Since Microsoft 
does not take advantage of a standardize virtual file system like Linux the operating 
system must know each filesystem and how to specifically talk to the media. Windows 
achieves this with an advanced \textit{I/O Manager} that in turn uses drivers just 
like the Linux kernel to perform the actually operations on to the filesystem~\cite{WI16}. 

Windows either at boot or when a new volume is mounted first tries to identify what 
type of filesytem the volume contains. Every Widows supported filesystem contains 
the information need for the \textit{FSD} (Filesystem Driver) in the volumes boot sector.
If the filesystem format is unrecognized or 
boot sector has been corrupt rendering the information Windows uses to identify the 
volumes filesystem unreadable Windows will default to the \textit{RAW} \textit{FSD}. When Windows 
declares a filesystem as \textit{RAW} the user is prompted to see if they would 
like to format the volume. If the boot sector is intact and Windows identifies the 
filesystem as a supported type Windows creates a \textit{Device Object} that the 
operating system will use to map the requested I/O operations to the physical 
media. After a \textit{FSD} claims a volume all of the I/O operations to that volume is 
passed trough the \textit{FSD}. All of the I/O operations are mapped from the \textit{Device Object} 
by the \textit{VPB} (Volume Parameter Block) to the volumes responsible \textit{FSD}.
When Windows mounts a volume and assigns a drive letter to it, it is really just 
a symbolic link to the \textit{Device Object}~\cite{WI26}.

\subsection*{Comparison}

\noindent \textbf{Similarities}\\
\begin{itemize}
    \item Read information from the MBR(master Boot Record).
    \item Each filesystem gets a file descriptor mapped to unique location.
    \item Need a filesystem driver to write to actual volume.
\end{itemize}
\noindent \textbf{Differences}\\
\begin{itemize}
    \item Windows kernel needs to be aware of what the underlaying filesystem is.
    \item Volumes in Windows get assigned letters, Volumes are assigned devices 
    with numbers in Linux.
\end{itemize}

\bibliographystyle{plain}
\bibliography{ref}

\end{document}
