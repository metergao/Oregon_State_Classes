\documentclass[letterpaper,10pt,notitlepage,fleqn]{article}

\usepackage{nopageno} %gets rid of page numbers
\usepackage{alltt}                                           
\usepackage{float}
\usepackage{color}
\usepackage{url}
\usepackage{balance}
\usepackage[TABBOTCAP, tight]{subfigure}
\usepackage{enumitem}
\usepackage{pstricks, pst-node}
\usepackage{geometry}
\geometry{textheight=9in, textwidth=6.5in} %sets 1" margins 
\newcommand{\cred}[1]{{\color{red}#1}} %command to change font to red
\newcommand{\cblue}[1]{{\color{blue}#1}} % ...blue
\usepackage{hyperref}
\usepackage{textcomp}
\usepackage{listings}

\def\name{Lawrence Chau}

\parindent = 0.0 in
\parskip = 0.2 in

\title{Project 2 Write Up}
\author{Lawrence Chau}

\begin{document}
\maketitle
\hrule

\section*{What do you think the main point of this assignment is?}
I think the main point of these assignments so far is to help us continue get more and comfortable with the Linux kernel source code. This includes getting familiar with the location and content of specific files, like the I/O scheduler in this case. This was not something we could have easily jumped into without reading the book and understanding how schedulers behave. The frustration of having to recompile such a large image over and over again, no matter how small
the change may be, helped us develop a firm understanding behind the workings of the scheduler. 

\section*{How did you personally approach the problem? Design decisions, algorithm, etc.}
In order to approach this problem, we had to first understand the Noop scheduler, Shortest Seek Time First schedulers, and how they prioritize I/O requests. From a bit of reading, I learned that the Noop scheduler takes the next request from the beginning of the queue and dispatches it, and appends requests to the back of the queue. It works almost like a FIFO. Shortest Seek Time First takes the position of the head, the position of the request, and the direction of the head in order to find the closest file to dispatch. Once we understood this, we were able to start reading the noop-iosched.c.

Overall, the noop-iosched.c contained most of the functions we needed to implement our scheduler. We really only needed to change how it handled the dispatches. First, the scheduler makes sure the queue is not empty and that there are values that it may compare. If it is going forward and the request is ahead of the list head, then only those values will be compared until the smallest absolute distance between the two are found. If you are going backwards, then you are only considering and comparing against the values behind the head. After the smallest absolute values is found, that I/O process is dispatched. Afterwards, the head is changed either to the front or bottom of the sector that it made the dispatch on for future dispatches.

\section*{How did you ensure your solution was correct? Testing details, for instance.}
To ensure that our solution was correct, we used the \textit{printk} function to have the kernel print messages telling us the status of the queue, where's it is dispatching, itterating, and when it changes directions. We then copied the Linux kernel folder over to the desktop and observe the behavior until we deemed it to be acceptable. There was really no way for us to actually confirm whether or not it was working but to make sure the head was changing direction every now and then.

\section*{What did you learn?}
I learned quite a bit about different IO schedulers, how to implement a different scheduler, and compile kernel code. It surprised me that the scheduler was only composed of a handful of functions, since I originally expected something almost impossible to understand.

\end{document}
