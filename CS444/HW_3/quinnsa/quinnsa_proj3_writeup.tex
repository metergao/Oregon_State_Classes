\documentclass[letterpaper,10pt,notitlepage,fleqn]{article}

\usepackage{nopageno} %gets rid of page numbers
\usepackage{alltt}                                           
\usepackage{float}
\usepackage{color}
\usepackage{url}
\usepackage{balance}
\usepackage[TABBOTCAP, tight]{subfigure}
\usepackage{enumitem}
\usepackage{pstricks, pst-node}
\usepackage{geometry}
\geometry{textheight=9in, textwidth=6.5in} %sets 1" margins 
\newcommand{\cred}[1]{{\color{red}#1}} %command to change font to red
\newcommand{\cblue}[1]{{\color{blue}#1}} % ...blue
\usepackage{hyperref}
\usepackage{textcomp}
\usepackage{listings}

\def\name{Sam Quinn}

\parindent = 0.0 in
\parskip = 0.2 in

\title{Project 3 Write Up}
\author{Sam Quinn}

\begin{document}
\maketitle
\hrule

\section*{What do you think the main point of this assignment is?}
    I think the main point of this assignment was to incorporate existing source 
    code provided by the Linux Kernel in to a new function. There is a lot of reusable 
    code within the Linux kernel and when ever possible you should use that instead 
    of trying to implement it yourself. I think that since there was little documentation 
    on the \textit{Linux Crypto API} it made us have to analyze the source code 
    directly which is not an easy task.

\section*{How did you personally approach the problem? Design decisions, algorithm, etc.}
    I first wanted to implement a \textit{RAM Disk} without encryption and then implement 
    the encryption portion later. After doing some research on \textit{RAM Disks} 
    I found that there is a command that you can create a \textit{RAM Drive} but 
    to have it \textbf{Encrypt} and \textbf{Decrypt} in the background without any 
    configuring I needed to create a Driver within the Kernel. I read the LDD3 section 
    on creating a Linux \textit{RAM Disk} driver and began to examine the source 
    code provided. I continued my research and found a blog post by ``Pat Paterson'' 
    which took the same source code provided by LDD3 book and fixed many of the 
    compile errors that would occur with Linux Kernel 2.6 and up. I took his code 
    which looks fairly identical to the one out of the book and implemented it in 
    to our Kernel. Once it booted and I figured out how to mount it we began the 
    wild goose chase of figuring out how to implement the Linux crypto API. We came 
    across a few example which described the process allocate cipher, initialize 
    key, encrypt/decrypt, and then free the cipher. After we got that information 
    we began to implement it into our RAM Disk. While every example we found online 
    about the ciphers used ``AES'' as their encryption method. Trying to stand out 
    we at first tried to implement the ``Blowfish'' cipher, which ended up not working.
    We eventually got the cipher to encrypt and decrypt one byte at a time which 
    requires us to loop through our data to encrypt and decrypt.

\section*{How did you ensure your solution was correct? Testing details, for instance.}
    We implemented a few printk statements that would display the data as unsigned 
    chars before and after the encryption. It was quite easy to see that the data 
    which was already quite foreign because it was all numbers was getting more 
    jumbled after the encryption process. Even when the data being passed through 
    the cipher are all zeros you can see that the output became something else.

\section*{What did you learn?}
    This project interested me a lot. I am currently going in to the security field 
    of computer science and this was the fist time that I have worked with encryption 
    at this low of a level. I think that I might even try to implement the same 
    concept to one of my extraneous drive in my personal computer. I learned 
    how to take advantage of reusable code within the Linux Kernel as well as 
    get first hand experience in the exact field I hope to be apart of after school.
\end{document}
