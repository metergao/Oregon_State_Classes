\documentclass[letterpaper,10pt,titlepage,fleqn]{article}

%example of setting the fleqn parameter to the article class -- the below sets the offset from flush left (fl)
\setlength{\mathindent}{1cm}

\usepackage{graphicx}                                        

\usepackage{amssymb}                                         
\usepackage{amsmath}                                         
\usepackage{amsthm}                                          
\usepackage{nopageno}
\usepackage{alltt}                                           
\usepackage{float}
\usepackage{color}
\usepackage{wasysym}
\usepackage{url}

\usepackage{balance}
\usepackage[TABBOTCAP, tight]{subfigure}
\usepackage{enumitem}

\usepackage{pstricks, pst-node}

%the following sets the geometry of the page
\usepackage{geometry}
\geometry{textheight=9in, textwidth=6.5in}

% random comment

\newcommand{\cred}[1]{{\color{red}#1}}
\newcommand{\cblue}[1]{{\color{blue}#1}}

\usepackage{hyperref}

\usepackage{textcomp}
\usepackage{listings}

\def\name{Sam Quinn}

%% The following metadata will show up in the PDF properties
\hypersetup{
  colorlinks = true,
  urlcolor = black,
  pdfauthor = {\name},
  pdfkeywords = {cs311 ``operating systems'' files filesystem I/O},
  pdftitle = {CS 311 Project},
  pdfsubject = {CS 311 Project},
  pdfpagemode = UseNone
}

\parindent = 0.0 in
\parskip = 0.2 in

\begin{document}

\section*{Questions}

\hrule

\begin{enumerate}
\item Describe at least 2 ways of transferring files from a remote server to a local machine.
\begin{itemize}
\item FTP(File Transfer Protocol) and SSH(Secure Shell) are two very common ways to retrieve files from a remote server.
\end{itemize}
\item What are revision control systems? Why are they useful? Explain how to create a subversion or git repository on os-class (and create it, while you're at it).
\begin{itemize}
\item Revision control systems help identify which version is currently in use with things like a version number. Revision control systems can help in debugging and updating because if a know bug is in a file and the file hasn’t been updated you can try the updated file first.
\item To create a git repository on the OS-Class server you just use the command ``git init” to create an empty repository.
\end{itemize}
\item What is the difference between redirecting and piping? Describe each.
\begin{itemize}
\item Redirecting sends and receives data from files or streams.
\item Piping sends and receives data to and from other programs
\end{itemize}
\item What is make, and how is it useful?
\begin{itemize}
\item Make is useful for compiling multiple programs or compiling certain programs with default variables and flags. By typing ``Make" into the command line you could save yourself from typing multiple command line entries into the prompt.
\end{itemize}
\item Describe, in detail, the syntax of a make file.
\begin{itemize}
\item The syntax for calling a function in a makefile is just the function name. Functions declared with \textlangle filename\textrangle : followed by a colon.\\
The code contained in each function of the makefile will be imputed directly into the command prompt. 
\end{itemize}
\item Give a find command that will run the file command on every regular file (not directories!) within the current filesystem subtree.
\begin{itemize}
\item find . -name 'Filename*' -type f\\
Find triggers the find function -name signify that you are searching for a specific thing make sure you use the single quotes around the filename. The * is explaining that that the name before the * is not complete and doesn't have a file extension, it will return every file with those letters in that order to you.
\end{itemize}
\end{enumerate}



\end{document}