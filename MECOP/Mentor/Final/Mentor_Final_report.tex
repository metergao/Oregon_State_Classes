\documentclass[letterpaper,10pt,notitlepage,fleqn]{article}

%\usepackage{nopageno} %gets rid of page numbers
\usepackage{alltt}                                           
\usepackage{float}
\usepackage{color}
\usepackage{indentfirst}
\usepackage{url}
\usepackage{balance}
\usepackage[TABBOTCAP, tight]{subfigure}
\usepackage{enumitem}
\usepackage{pstricks, pst-node}
\usepackage{geometry}
\geometry{textheight=9in, textwidth=6.5in} %sets 1" margins 
\newcommand{\cred}[1]{{\color{red}#1}} %command to change font to red
\newcommand{\cblue}[1]{{\color{blue}#1}} % ...blue
\usepackage{hyperref}
\usepackage{textcomp}
\usepackage{listings}
\usepackage{graphicx}

\def\name{Sam Quinn}

\parindent = 0.4444 in
\parskip = 0.2 in

\begin{document}
\begin{titlepage}
\vspace*{\fill}

\newcommand{\HRule}{\rule{\linewidth}{0.5mm}} % Defines a new command for the horizontal lines, change thickness here

\center % Center everything on the page

%----------------------------------------------------------------------------------------
%TITLE SECTI   ON
%----------------------------------------------------------------------------------------

\includegraphics[scale=.17]{258_mentor_logo.eps}
\HRule \\[0.4cm]
{ \huge \bfseries Mentor Graphics Report}\\[0.4cm] % Title of your document

%----------------------------------------------------------------------------------------
%HEADING SECTIONS
%----------------------------------------------------------------------------------------

\textsc{\LARGE Sam Quinn}\\[0.5cm] % Name of your university/college
\textsc{\Large Mecop Internship \#1}\\[0.5cm] % Major heading such as course name
\textsc{\large Spring/Summer 2014}\\[0.5cm] % Minor heading such as course title


\HRule \\[1.5cm]
%----------------------------------------------------------------------------------------
%AUTHOR SECTION
%------------------------------------ ----------------------------------------------------

\begin{minipage}{0.4\textwidth}
\begin{flushleft} \large
\emph{Intern:}\\
        \textbf{Sam \textsc{Quinn}} \\ % Your name
        {\small Computer Science}
        \end{flushleft}
        \end{minipage}
        ~
        \begin{minipage}{0.4\textwidth}
        \begin{flushright} \large
        \emph{Mentors:} \\
            \textbf{Mark \textsc{Gaidos}} \\ % Supervisor's Name
            {\small Mark\_Gaidos@mentor.com \\
            (503)685-5151} \\
            
            \textbf{Sam \textsc{McDowell}}
            {\small Sam\_McDowell@mentor.com \\
            (503)685-1079}
            \end{flushright}
            \end{minipage}\\[3cm]

            % If you don't want a supervisor, uncomment the two lines below and remove the section above
            %\Large \emph{Author:}\\
                %John \textsc{Smith}\\[3cm] % Your name

                %----------------------------------------------------------------------------------------
                %DATE SECTION
                %-----------------    -----------------------------------------------------------------------

{\large \today}\\[3cm] % Date, change the \today to a set date if you want to be precise

%----------------------------------------------------------------------------------------
%LOGO SECTION
%------   ----------------------------------------------------------------------------------

%\includegraphics{Logo}\\[1cm] % Include a department/university logo - this will require the graphicx package

%----------------------------------------------------------------------------------------

\vfill % Fill the rest of the page with whitespace



\end{titlepage}

\tableofcontents
\newpage

\section{Introduction}
\indent For my first MECOP internship I was fortunate to be placed at Mentor Graphics. This 
was my first time being immersed into a professional atmosphere in the field that 
I am studying and I really enjoyed it. I am studying computer science and even thought 
Mentor Graphics is more of an Electrical Engineering company, I met multiple people 
here that were also computer scientists. It was good for me to learn the basics 
of electrical engineering, I now feel more well rounded and can understand the hardware 
behind the software in more detail.

\section{Mentor Graphics}

\subsection{What does Mentor Graphics do?}
\indent When I was first placed at Mentor Graphics I had not heard much about them. I initially 
though Mentor Graphics made computer Graphics Cards or something, but soon learned 
that was not the case. Mentor Graphics is one of the leading EDA (Electronic Design Automation)
companies. Mentor Graphics has products that help every step of creating printed 
circuit boards and integrated circuits. Customers often use Mentor Graphic's products 
during the design process to ensure that their boards and chips will function properly. 

\subsection{Calibre}
\indent During my internship at Mentor Graphics I was apart of the Calibre team which is 
an IC physical verification tool. Calibre runs sets of tests that check 
the IC layout in pre\-production stages. My time was mostly allocated to help with 
a new feature of the Calibre DFM tool that creates ``dummy fill'' shapes within 
the IC layout to allow the cutting to be more precise. 

\subsection{Mentor/Supervisors}
\indent I worked with a lot of individuals during my time at Mentor Graphics but mostly with 
my Mentors Mark Gaidos and Sam McDowell. Mark is a Staff engineer and has been working 
at Mentor Graphics for 27 years and has been a great supervisor. During my work with 
Calibre's DFM FILL WRAP function I worked very close to Sam McDowell. Sam was Mark's 
first intern, which was very cool to see that they are still working together. Sam 
was my go to person for questions and concerns for most of my internship. Sam and 
I also played on the Calibre softball team together, which was a fun way to connect 
outside of the work environment.

\section{Projects}

\subsection{Tinstall}
\indent Tinstall was one of the first projects I worked on. Tinstall's purpose is to create 
a more uniform, organized, and easy to maintain directory for internally created 
tools and utilities. Tinstall reads from a configuration file to get instructions 
on which files to install and where. Each file must be CVS revision controlled and 
up to date or else Tinstall will error out. Tinstall must verify that the user has 
permissions to install to the requested directory. Tinstall will create necessary folders and only overwrite 
a file if it has been updated. If Tinstall determines that everything checks 
out, it will install all of the files to the correct directories and create desired 
symbolic links. In the case of an installation failing, Tinstall will automatically restore the 
contents of the install directory to the state before Tinstall attempted the install. After a successful 
install, Tinstall creates a ``bill of materials'' file and updates the log with the 
user who executed the installed.
\\
\indent This was the first time I have been introduced to the programing language Tcl. 
Since Eric Martinson had begun to create this tool before me, there were 
some files in place that I studied to become familiar with Tcl. It was really 
nice to have some skeleton code in place before I started, and to view how a professional
laid out the base of the program. Mimicking the coding style helped me create a well designed 
tool. I ended up liking Tickle a lot, even though many people seem to dislike it. 
\\
\indent One of the biggest takeaways from this project was the importance of error checking.
In my past projects that I completed for school and personal use, I often 
never implemented error logs or error checking. Studying the code before beginning 
my work on this project, I noticed that there were comments and error checks everywhere.
Since I mimicked Eric's coding style, I too used an abundance of comments and checks.
I soon came to the realization of how much easier it is to find and 
take care of bugs when you know where they come from. \\


\textbf{Things Learned:}
\begin{itemize} 
        \item Tcl
        \item Unix Permissions
        \item Semaphores
        \item Following design guidelines
        \item Fail-Safe programing techniques
    \end{itemize}

\subsection{DFM FILL WRAP QA}
\indent The majority of my time spent at Mentor Graphics was helping Sam with testing and 
debugging the new DFM Fill Wrap feature. There was an existing DFM Fill Wrap in place 
when I started but with how much smaller the IC layouts are getting it was causing 
problems. Developers decided to make a new Wrap function that can handle smaller 
layouts like 10nm designs. It was very interesting to be apart of this new product 
and to understand the need for it.
\\
\indent Eugene explained to me that the reason the old Wrap Fill will no longer 
work is that designs now require precision cuts that are smaller than the light wave used for 
optical cutting. This problem can be solved passing 
the light through a viscous chemical that concentrates the light into multiple 
parallel straight lines. This is the root of the problem with generating fill shapes. 
The old Fill Wrap would place dummy fill shapes in a bi-directional 
manner, since now the cuts are only created in parallel straight lines there 
cannot be bi-directional fill shapes only uni-directional. 
\\
\indent  Sam was really good about including me in on all of the Wrap related 
emails and always invited me to the meetings. I created many test cases 
and felt like I was a real addition to the team. Also the fact that we are 
now creating microchips that need precision smaller than the width of a light wave 
fascinates me, and I am very fortunate to have worked on a bleeding edge product. 

\textbf{Things Learned:}
\begin{itemize} 
        \item How IC are cut using light and chemicals
        \item CVS Repositories
        \item Creating test cases
        \item Working with a QA team
        \item How to use Calibre software
    \end{itemize}

\subsection{PERC QA}
\indent With about a month left in my internship, I was asked to help with 
PERC QA. PERC is a part of the LVS tool, which compares a finished layout with the schematic 
to ensure that the physical implementation of the circuit matches its logical definition 
thus the name Layout Vs. Schematic. Specifically PERC is Programmable Electronic Rule 
Checking that verifies layouts with advanced electrostatic discharge checks.
\\
\indent I was called on board to help out with PERC to help gather information about around 
400 tests that fail when ran in forced parallel mode. The PERC test suite was built 
much differently than that of the DFM tests I was used to. The PERC test utilized an intricate 
and intelligent harness that would analyze each test case and figure out how to run 
it automatically. It was fairly difficult to get caught up on the PERC QA logic 
because of how different it was from DFM. I will be creating a dictionary of all 
the PERC built-ins that will be correlated to existing tests that are failing. 
Special analysis and repair efforts can be stream lined to built-ins with low 
coverage or failing patterns.

\textbf{Things Learned:}
\begin{itemize} 
        \item More Tcl
        \item Using a harness to run test cases
        \item How to creating a dictionary for debugging purposes
        \item How to analyze a core file after a core dump
    \end{itemize}

\section{Minor Projects}

\subsection{TWiki Upgrade}
\indent During my down time while working with Sam debugging the Fill Wrap feature I had 
the opportunity to gain knowledge in website hosting. I was asked to investigate 
what needed to be done to upgrade DFM QA's internal Wiki site. The DFM QA Wiki 
site was created with TWiki and open source wiki interface and hosted locally at Mentor 
Graphics. Before diving head first into this project, I had no prior experience working 
with Apache, CGI scripts, or really any web tools for that matter, but I was very 
excited to learn. \\
\indent I spent about a week creating an exact replica of the DFM QA TWiki site on my local 
machine with the updated version. I ran into a lot of problems with permissions 
and how Apache can own files that even root can't access. With help from Sam and 
ICV BIG, I eventually got a working copy running on my computer. Since the TWiki 
was accessed by all of the DFM QA team, there was too much information for me to 
verify everything copied correctly during the upgrade. The way Sam and I tested 
everything was correct, was we did a live switch and allowed all of the team to 
poke around the new version to check that all of their paths and normal day links 
continued to work. After we got our good results back from the team, we copied my 
local copy to the permanent server location. \\

\textbf{Things Learned:}
    \begin{itemize} 
        \item Apache configuration
        \item Unix Permissions
        \item CGI scripts
        \item General web hosting 
    \end{itemize}

\subsection{Test Suite Search Tool}
\indent After working with Dave in my Tinstall project, Dave approached me with a problem 
that he thought I could help with. Dave's problem was that when he finds a bug within 
the test suite he often times did a series of greps and pipes to narrow down his 
search. Since there wasn't really any built-in UNIX or inside tool that could search 
for multiple terms, I decided to help him with this. \\ 
\indent Dave gave me the freedom to use whatever language I thought best, since I was writing 
this program from scratch. After enjoying learning TCL so much in the Tinstall 
program, I decided to learn a new language and went with Perl for the code and Tk 
for the GUI. Dave thought Perl and Tk were a good choice and after explaining to 
him that I have never programmed in Perl before he showed me our company library, 
which was a great resource in later projects.

\textbf{Things Learned:}
\begin{itemize} 
        \item Tk
        \item Perl
        \item Making a GUI
        \item Grep to a higher degree
    \end{itemize}

\section{Conclusion}
\subsection{What I learned}
\indent During my 6 month internship I learned a comparable amount of degree related knowledge
than I would have at school. The main difference was that the things I learned while 
Working at Mentor Graphics no class could teach. One of my most exciting things 
I learned during my internship is how to be efficient. I consider myself 100\% more 
comfortable in a UNIX environment. I believe that I will be a much better programmer 
and will be able to produce a higher quality of work from the small tips and tricks 
that were taught to me during my time at Mentor Graphics. The real world work environment seems 
far less scary to me now that I have had a slight taste of it. It was very neat 
to see how my background knowledge from school got integrated into the work place.
I would often times recall information from previous classes that would help me 
solve problems I faced during my internship. 
\\
\subsection{How the internship benefited me}
\indent This internship has greatly benefited me in many ways. One of the greatest 
benefits I will takeaway from this internship is knowing that I can work in the 
industry. It was great to see that I was working on real projects that full time 
employees were working on and helped solve unsolved problems. I liked working on 
nonstandard problems in contrast to how at school I felt like I was finishing 
a homework assignment that I know every other CS student has solved before me. 
\\
\subsection{How I benefited Mentor Graphics}
\indent I really didn't know how important my contributions were to Mentor Graphics until 
I was talking to  one of the employees about what I was working on with the new Wrap function and he said "You 
are working on probably the most important update in the upcoming release". After 
I heard this I truly felt like Mentor Graphics valued my work and that I was actually
making a difference. I created more than 30 test cases, upgraded a site most people 
on the QA team access daily, and helped find which functions that need to be tested.\\ 
\subsection{Presentations}
\indent After finishing Mentor Graphic's internal tool, Tinstall, I was asked to present to 
a majority of the QA employees explaining my tool. This presentation took place via 
telecommunication with people viewing from Mentor Graphics locations all over the 
world.
\end{document}

